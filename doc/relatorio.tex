\documentclass{article}
\usepackage[T1]{fontenc}
\usepackage[utf8]{inputenc}
\usepackage[brazil]{babel}

\usepackage{amsmath}
\usepackage{tikz}
\usetikzlibrary{positioning}
\usetikzlibrary{shapes}

\begin{document}

\title{
    INE5430 --- Inteligência Artificial \\
    NeuroTruck --- Rede Neural Estacionadora de Caminhões
}
\author{
    \begin{tabular}{r l}
        Lucas Berri Cristofolini & 12100757 \\
        Tiago Royer & 12100776 \\
        Wagner Fernando Gascho & 12100779
    \end{tabular}
}

\maketitle

\section{Introdução}

O objetivo deste trabalho era construir uma rede neural
que fosse capaz de estacionar um caminhão,
a partir do sistema fuzzy que estaciona o caminhão
feito para o trabalho anterior.

Utilizamos a biblioteca de redes neurais OpenNN (\verb"http://opennn.cimne.com/").

\section{Implementação}

Colocamos o sistema fuzzy do trabalho anterior para dirigir o caminhão
em diversos pontos iniciais.
Anotamos as quádruplas ordenadas com as entradas e a saída do sistema fuzzy
no arquivo \verb"data.dat";
e então fizemos a rede neural aprender este conjunto de dados.

O programa que faz o treinamento é \verb"NeuralTrainer",
de \verb"NeuralTrainer.cpp"; este programa lê \verb"data.dat"
e escreve o resultado do treinamento em \verb"neural_network.xml".
Os testes indicaram que $25$ neurônios internos são suficientes
para que a rede aprenda o conjunto de dados.
\footnote{
    Um comportamento interessante foi observado quando deixamos a rede
    com apenas $7$ neurônios internos.
    Estranhamente, o ``neurotruck''
    sempre tentava se jogar contra a parede da esquerda,
    não importando onde fosse colocado...
}

O programa \verb"main" lê \verb"neural_network.xml"
e usa os pesos gravados neste arquivo para se comunicar com o servidor.

No geral, o desempenho da rede neural foi um pouco melhor que o do FuzzyTruck;
acreditamos que este comportamento
seja devido às posições iniciais do caminhão que foram utilizadas
para gravar \verb"data.dat"
--- isto é, a rede neural aprendeu as ``melhores partes'' do FuzzyTruck.

\end{document}
